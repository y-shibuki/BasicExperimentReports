\documentclass[11pt]{jsarticle}
\usepackage{amsmath}
\begin{document}
\title{ナイロンの合成}
\author{1X18D078-4 \\ 吉田 陽向}
\maketitle
\newpage
%------ここまで表紙------%
\section{目的}
代表的な合成繊維であるナイロンは、アミド結合で連結された高分子である。
本実験では二種類のモノマーの縮合重合および環状モノマーの開環重合によりナイロン66およびナイロン6を合成し、物理的、化学的性質を調べる。
ナイロン66は膜、糸、カプセルの3つの異なる形状を界面重合で得る。
膜を用いて赤外分光法によりアミド結合の生成を確認し、カプセルでは半透膜としての性質を調べる。
ナイロン6については引張強度を測定する。
これらの実験を通して化学薬品、ガラス器具、バーナーなどの安全な取り扱い、赤外分光光度計の原理および操作を習得する。
\subsection{界面重合によるナイロン66の合成}
アジピン酸と1,6-シクロヘキサンの縮合重合により世界初の合成繊維としてナイロン66が合成された。
本実験では二塩化アジポイル(アジピン酸ジクロイド)と1,6-ヘキサンジアミンの反応によってナイロン66を合成する。
二塩化アジポイルはアジピン酸よりも反応性が高く、室温でも1,6-ヘキサンジアミンと速やかに反応し、
二当量の塩化水素を脱離してナイロン66を生成する。
実際には1,6-ヘキサンジアミン水溶液と二塩化アジポイルのヘキサン溶液の界面で反応が進行するので界面重合という。
生じた塩化水素は1,6-ヘキサンジアミンとの反応を妨げるために加えた炭酸水素ナトリウムと中和され、反応はナイロン66を生成する方向に進む。
\subsection{ナイロン66カプセル}
混じりあわない二液の界面でナイロンが生成することを応用して、液滴の界面でナイロンを薄膜として生成させるとカプセルを作製できる。
カプセル内には種々の物質を取り込ませることが可能であり、カプセルは半透膜の性質を有する。
本実験では内包させる物質として青色色素・トリパンブルーを用いる。
二塩化アジポイルのヘキサン溶液にトリパンブルーを溶かした1,6-ヘキサンジアミン水溶液を滴下すると、
二液の界面で反応が進行しトリパンブルーを内包したナイロンカプセルが作製できる。
\subsection{開環重合によるナイロン6の合成}
$\epsilon$-カプロラクタムは7員環化合物で歪みがあるため、加熱下で重合促進剤が存在すると式のように環が開いて相互に重合しナイロン6となる。
本実験では短時間で反応を進めるためにナトリウム-tert-ブトキシドを塩基として使用し、
さらに反応開始の促進剤としてN-アセチル-$\epsilon$-カプロラクタムを微量加えて行う。
加熱すると混合した試薬が溶融し、粘土が増して高分子化合物が生成する。
ガラス棒の先に付けて引っ張ると、丈夫で光沢のある繊維を紡糸できる。
紡糸したナイロン6については引張強度を測定するが、本実験で得られたものはアルカリ成分と低重合物を含んでいるために本来の強度は示さない。
\subsection{赤外分光法}
赤外分光法は赤外線の持つエネルギーを利用して、物質の同定、定量、構造解析を行う機器分析の一つである。
赤外線を物質に照射するとン分子の双極子モーメントを変化させる振動が赤外線を吸収する。
この九州は分子を構成する結合や分子間相互作用に起因して原子団に特有である。
波長$\lambda$の逆数を波数$\tilde{\nu}$(単位:$\mathrm{cm^{-1}}$)とすると、式のようにエネルギーを波数$\tilde{\nu}$で表すことができる。
赤外線は波長$2.5~25 \mathrm{\mu m}$の電磁波であるため、エネルギーは$4000~400 \mathrm{cm^{-1}}$の波数$\tilde{\nu}$に相当する。
\begin{equation}
    E=hc\tilde{\nu}
\end{equation}
測定では物質に強度$I_0$の赤外光を入射して、吸収されずに透過してきた光の強度$I$を検出し波数に対して式で定義される透過率$T$(単位:$\mathrm{\%}$)
をプロットしたグラフ(赤外線吸収スペクトル)を得る。
\begin{equation}
    T=\frac{I}{I_0} \times 100
\end{equation}
赤外線を吸収すると透過率$T$は減少する。
赤外線吸収スペクトルの吸収帯の波数$\tilde{\nu}$を既知のグループ振動の波数$\tilde{\nu}$と照合することで、結合や官能基を同定できる。
本実験では合成したナイロン66膜の赤外線吸収スペクトルを測定し、アミド結合を確認する。
アミド結合に由来するN-H伸縮振動、C=O伸縮振動、N-H変角振動による吸収帯が観測されることで、ナイロンが生成した実験的証拠となる。
\end{document}