\documentclass[11pt]{jsarticle}
\begin{document}
\title{カフェインの抽出}
\author{1X18D078-4 \\ 吉田 陽向}
\maketitle
\newpage
%------ここまで表紙------%
\section{目的}
日常的な飲料であるコーヒーからカフェインを分離し、得られた物質がカフェインであることを薄層クロマトグラフィー法によって確認する。
また実験を通して科学実験における基本技術と器具や薬品の取り扱い方を身に付ける。
\section{理論・原理}
\subsection{カフェイン}
テインとも呼ばれ、正式名称は1,3,7-トリメチルキサンチンである。
一般に一水和物であり光沢のある白色結晶として存在し、昇華性を示す(178℃)。
クロロホルムによく溶け、酢酸エチルや水、アルコールにも可溶である。
しかし、石油エーテルにはほぼ不溶である。
\subsection{混合物の分離法}
\subsubsection{抽出}
混合物からある特定の物質のみを溶媒に溶かして分離する操作。
また互いに混ざらない液体のいずれにも溶ける物質を一方の液体のほうに移動させて抽出する方法もある。
\subsubsection{蒸留}
成分の沸点の違いを利用することで分離する方法。
特に固体物質の溶けた溶液から溶媒だけを取り除く操作を「溶媒留去」、液体同士の混合物においてそれぞれを分離することを「分留」という。
\subsubsection{再結晶}
物質の溶解度の違いを利用して分離する方法である。
溶解度は溶かす溶媒や溶質の組み合わせだけでなく、温度によっても著しく変化する。
\subsubsection{ろ過}
固体と液体の混合物においてその固体が液体に溶けていない状態の時にろ紙などを利用して分離する操作。
\subsubsection{クロマトグラフィー}
個々の物質の移動速度の違いを用いて混合物を物理的に分離する技術である。
\subsubsection{昇華法}
固体が液体を経ずに直接気体になる変化を昇華といい、この性質を利用して分離する方法を「昇華法」という。
昇華性を有する物質としてヨウ素、ナフタレン、カフェインがある。
\subsection{薄層クロマトグラフィー}
まず分離したい混合物を固定相が塗布された薄層板の下部にスポットする。


\end{document}