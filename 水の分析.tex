\documentclass[11pt]{jsarticle}
\usepackage{amsmath}
\begin{document}
\title{水の分析}
\author{1X18D078-4 \\ 吉田 陽向}
\maketitle
\newpage
%------ここまで表紙------%
\section{目的}
未知試料中の鉄の定量をJIS公定法であるフェナントロリン吸光光度法により行い、吸光光度法の原理を理解する。
実験を通して環境保全の重要性および水質基準等の法規を知り、
環境保全上健全な水環境の維持、飲料水の安全性のために分析化学の知識と実践が重要であることを理解する。
\section{理論}
\subsection{吸光光度法}
試料中の特定成分による光吸収の度合いを測定して、その特定成分を定量する方法。主に溶液を対象とする。
資料溶液に波長が単一の単色光が入射すると容器の表面での反射がおこるが、多くの光は透過する。
吸光光度法と本質的に関係があるのは光吸収(図1)である。
光の吸収率は入射する光子のエネルギーが電子の遷移にかかわるエネルギー準位の間隔に等しい時に最大となる。
\subsection{Lambert-Beerの法則}
吸光光度法の基本となる吸収の光度はLambert-Beerの法則に従う。
図1においてセルの厚さを$l$とすると吸光度$A$は溶液の濃度$c$と暑さ$l$に比例する。
溶液へ入射したときの光の強度を$I_0$とし溶液を透過した時の光の強度を$I$とすると、次の式で表すことができる。
\begin{equation}
    \frac{I}{I_0}=T (Tは透過度)
\end{equation}
溶液の厚さが増すと透過光の強度$I$は指数関数的に減少する。
このことは、濃度$c$が一定の場合透過度$T$の逆数の対数(これを吸光度といい量記号$A$で表す)は、溶液の厚さ$l$に比例することを意味している。
\begin{gather}
    A=\log \frac{1}{T}=-\log T=\log \frac{I_0}{I}=acl \\
    a=\frac{A}{cl}
\end{gather}
同様に溶液の厚さ$l$を一定とすると、吸光度$A$は濃度$c$に比例する。$A$は比例定数であり吸光定数と呼ばれる。
\subsection{検量線}
呈色した濃度既知の試料の特定の波長における吸光度$A$と濃度$c$との関係を求めたものが検量線である。
標準溶液を用いて、その「各々の濃度$c$と吸光度$A$の関係」より検量線を作成し未知試料の吸光度を測定して検量線から濃度を求める検量線法を用いて
回帰分析を行うことにより、吸光係数$a$を求めることができる。
グラフの横軸に濃度($\mathrm{mg/L}$)をとり縦軸に吸光度をプロットする。このプロットでは直線$y=a'x$になると予想できるので、
最小二乗法により測定値を直線に回帰しその傾き$a'$から、溶液の厚さ$l$を加味することで、吸光係数$a$を求めることができる。
\begin{equation}
    a'=\frac{\sum x_{i}y_{i}}{\sum x_{i}^{2}}
    \Rightarrow a=a' \times \frac{1}{l}
\end{equation}
本実験では、濃度$c$の単位を$\mathrm{mg/L}$とし、厚さ$l$の単位を$\mathrm{cm}$として、吸光係数$a$を測定する。
この場合は、吸光係数$a$の単位を$\mathrm{L/(mg \cdot cm)}$である。
吸光係数$a$の値を解析的に決定した後、基本式を用いて吸光度から未知試料の濃度を求めることができる。
\subsection{空試験}
試料溶液を吸収セルに満たして光を通した場合、溶質だけでなく溶媒や吸収セルによっても光が吸収される。
したがって実際の測定ではこれらの媒体による光吸収について空試験を行うことで補正を行う必要がある。
空試験溶液とは目的成分を含まず、その他の試薬は試料溶液と全く同じように含んでいる溶液のことである。
その吸光度(空試験値という)を測定し、試料吸光度から差し引いた値(補正吸光度$A_c$)を用いて検量線を作成する。
\subsection{1,10-フェナントロリンによる鉄の定量}
1,10-フェナントロリンは図2(a)の構造を持ち、窒素原子の非共有電子対を金属イオンに供与して配位結合を生成するキレート試薬で無色の塩基である。
$Fe^{2+}$とは図2(b)に示した安定な八面体型錯体を形成し、この錯体は$510\,\mathrm{nm}$付近に吸収極大波長をもち橙赤色を示す。
従って分光光度計を用いて$510\,\mathrm{nm}$付近の吸収を測定することにより、最も感度よく$Fe^{2+}$を定量することができる。
橙赤色の生成は$Fe^{2+}$に特異的であるため、試料中の全鉄を定量するためにはあらかじめ還元剤を加えて$Fe^{3+}$を$Fe^{2+}$に還元しておく必要がある。

\end{document}

